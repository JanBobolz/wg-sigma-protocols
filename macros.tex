\newcommand{\serialize}{\texttt{Serialize}}
\newcommand{\deserialize}{\texttt{Deserialize}}
\newcommand{\frombytes}{\texttt{FromBytes}}
\newcommand{\statementnode}{\texttt{Statement}}
\newcommand{\statementleaf}{\texttt{StatementLeaf}}
\newcommand{\unsure}[1]{\textcolor{red}{#1}}
\newcommand{\simresponse}{\textsf{SimResp}}
\newcommand{\simcommitment}{\textsf{SimCom}}
\newcommand{\proverstatement}{\prover.\textsf{stmt}}
\newcommand{\andstmt}{\textsf{AND}}
\newcommand{\orstmt}{\textsf{OR}}
\newcommand{\seedlen}{\ensuremath{32B}}
\newcommand{\none}{\mathsf{None}}



\newcommand{\relation}{\mathsf{R}}
\newcommand{\lang}{\mathcal{L}}

%algorithms
\newcommand{\alg}{\mathsf{Alg}}

%sets
\newcommand{\CS}{\mathcal{C}}
\newcommand{\someset}{\mathcal{S}}


%variables
\newcommand{\st}{\mathsf{st}}
\newcommand{\ctx}{\mathsf{ctx}}


\newcommand{\minote}[1]{\todo[color=green!30, inline]{\textbf{Michele's note:} #1}}
\newcommand{\stnote}[1]{\todo[color=blue!30, inline]{\textbf{Stephan's note:} #1}}
\newcommand{\jannote}[1]{\todo[color=yellow!30, inline]{\textbf{Jan's note:} #1}}
\newcommand{\marynote}[1]{\todo[color=red!30, inline]{\textbf{Mary's note:} #1}}
\newcommand{\viscontinote}[1]{\todo[color=gray!30, inline]{\textbf{Visconti's note:} #1}}

\newcommand{\accept}{\pctrue}
\newcommand{\reject}{\pcfalse}

\newcommand{\curvectx}{\textsf{curve}}
\newcommand{\idctx}{\textsf{id}}
\newcommand{\generatorsctx}{\textsf{gens}}
\newcommand{\domsepctx}{\textsf{domsep}}

\newcommand{\batchprover}{\prover_{\operatorname{batch}}}
\newcommand{\batchverifier}{\verifier_{\operatorname{batch}}}
\newcommand{\shortprover}{\prover_{\operatorname{short}}}
\newcommand{\shortverifier}{\verifier_{\operatorname{short}}}

\newcommand{\mat}{\mathbf}
\renewcommand{\vec}{\mathbf}

\newcommand{\statement}{Y}
\newcommand{\witness}{w}
\newcommand{\commitment}{T}
\newcommand{\challenge}{c}
\newcommand{\response}{s}
\newcommand{\stag}{\tau}



\newcommand{\cmark}{\ding{51}}%
\newcommand{\xmark}{\ding{55}}%

% use protocol environment as follows:
% \begin{protocol}{CAPTION}{LABEL}{PLACEMENT}
%   ...
% \end{protocol}
% Currently, all arguments are MANDATORY!!!
\usepackage{newfloat}
\DeclareFloatingEnvironment{protocolFloat}

\newcommand{\templabel}{}% stores the label
\newcommand{\tempcaption}{}% stores the caption

\crefname{protocol}{Protocol}{Protocols}
\Crefname{protocol}{Protocol}{Protocols}

\usepackage{float}
\usepackage{fancybox}

\newenvironment{protocol}[3]
{\begin{protocolFloat}[#3]%   open floating environment
 \gdef\tempcaption{#1}%       store the caption so we can use it later
 \gdef\templabel{#2}%         store the label so we can use it later
 \begin{center}%              let's center everything
 \begin{Sbox}%                put a frame around the protocol
 \fontsize{8}{8}\selectfont%  reduce font size a little bit
}
{\end{Sbox}%                               end the frame
 \setlength{\fboxsep}{4pt}\fbox{\TheSbox}% fix the padding of framebox
 \\ \vspace{0.5em}%                        some vertical space to caption
 \centering%                               center the caption
 \refstepcounter{theorem}%                 use the same counter as for theorems
 {\bf Protocol~\thetheorem: }\tempcaption% caption
 \label[protocol]{\templabel}%             label
 \end{center}%                             end centering
 \end{protocolFloat}%                      end floating environment
}


\newcommand{\verylongrightarrow}[1]
     {\setlength{\unitlength}{.01in}
     \begin{picture}(#1,1) \put(0,0){\vector(1,0){#1}} \end{picture}}

\newcommand{\sendr}[2]{{\stackrel{\substack{#1}}
{\verylongrightarrow{#2}}}}


\newcommand{\verylongleftarrow}[1]
    {\setlength{\unitlength}{.01in}
    \begin{picture}(#1,1) \put(#1,0){\vector(-1,0){#1}} \end{picture}}

\newcommand{\sendl}[2]{{\stackrel{\substack{#1}}
{\verylongleftarrow{#2}}}}


\newcommand{\defeq}{\coloneqq}
\newcommand{\grgen}{\mathsf{GrGen}}

