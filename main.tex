% This is samplepaper.tex, a sample chapter demonstrating the
% LLNCS macro package for Springer Computer Science proceedings;
% Version 2.20 of 2017/10/04
%
\documentclass[runningheads]{llncs}
%
%
\usepackage{amsmath}
\usepackage{amssymb}
%\usepackage{amsthm}
% crypto notation
\usepackage[advantage,
            adversary,
            asymptotics,
            complexity,
            ff,
            lambda,
            mm,
            operators,
            primitives,
            probability,
            sets]{cryptocode}

%
\usepackage{mathtools}
%
\usepackage{todonotes}
%
\usepackage[%dvips,
    pdftex,
    pdftitle={Proposal: Sigma-protocols},
    pdfauthor={Stephan Krenn and Michele Orrù}, pdfpagelabels=true, %linktocpage=true,     backref=page,
    bookmarks,bookmarksopen,bookmarksdepth=3,
    breaklinks,colorlinks,citecolor=blue,linkcolor=blue
   ]{hyperref}
\usepackage[capitalize,nameinlink,sort]{cleveref}
%
\usepackage{graphicx}
% Used for displaying a sample figure. If possible, figure files should
% be included in EPS format.
%
% If you use the hyperref package, please uncomment the following line
% to display URLs in blue roman font according to Springer's eBook style:
\renewcommand\UrlFont{\color{blue}\rmfamily}

\usepackage{todonotes}

\newcommand{\HH}{\mathcal{H}}

\newcommand{\getsr}{\gets_{\$}}

\newcommand{\minote}[1]{\todo[color=green!30, inline]{\textbf{Michele's note:} #1}}

\begin{document}

\title{Proposal: $\Sigma$-protocols%
\thanks{This work has partially been funded by the European Union's Horizon 2020 framework programme under grant agreement no. 830929 (CyberSec4Europe).}}
%
\titlerunning{Proposal: $\Sigma$-protocols}

\author{Stephan Krenn\and
        Michele Orr\`u}

\authorrunning{S. Krenn and M. Orr\`u}

\institute{AIT Austrian Institute of Technology, Vienna, Austria \and
           University of California, Berkeley, United States}
%
\maketitle              % typeset the header of the contribution
%
\begin{abstract}
  We submit a proposal for the standardization of non-interactive sigma protocols in prime order groups, allowing for AND and OR composition, either in compact form (challenge, response) or batchable (commitment, response). We propose a selection of elliptic curves and hash functions suitable for implementations, together with a minimal API that should be consistent across implementations.

%\keywords{First keyword  \and Second keyword \and Another keyword.}
\end{abstract}

\section{Introduction}
Non-intearctive sigma protocols are zero-knowledge~\cite{GolMicRac89} proofs of knowledge~\cite{C:BelGol92} without SRS, proven secure in the random oracle model.
They have been introduced by Schnorr~\cite{JC:Schnorr91} and so far have there have been widespread implementations because of their simplicity, maturity, and versatility.
Recently, Lueks et al.~\cite{zksk},

Sigma protocols have already been employed for anonymous credentials, password-authenticated key exchange~\cite{jpake}, (ring, multi)  signatues, and other
cryptographic constructions. Solely in the past XX years, we recall the
\begin{itemize}
  \item DLEQ in some VOPRFs~\cite{cfrg-voprf}, Privacy pass~\cite{PoPETS:DGSTV18}
  \item AND+DLEQ anonymous credentials; (at least) AND+DLEQ: Signal~\cite{CCS:ChaPerZav20}, openings on Pedersen commitments, or plaintexts of ElGamal encryptions; DH tuple: Cryptography for \#metoo~\cite{PoPETS:KuyKraRab19}; Solidus~\cite{CCS:CZJKJS17}; ClaimChain~\cite{ClaimChain}
  \item OR for Ring Signatures: CryptoNote (Monero)~\cite{monero}, Mesh~\cite{PoPETS:AlTGon19};
  \item Signal, who's been using Algebraic MACs~\cite{CCS:ChaMeiZav14}; for group chats;
  \item Coconut~\cite{NDSS:SABMD19} with threshold Issuance Selective Disclosure Credentials with Applications to Distributed Ledgers;
   \item Anonymous Credentials Light~\cite{CCS:BalLys13}
\end{itemize}

A request for standardization has already been posted on the zk-proof community\footnote{\url{https://community.zkproof.org/t/standardizing-sigma-protocols/471/}} and received a fair amount of interest.

\subsection{Related Work}

\section{Background and Motivation}
\label{sec:motivation}
So far, there has been little literature meant to help cryptography engineers implement correctly Sigma protocols for arbitrary statements about discrete logs.
Despite Schnorr signatures and zero-knowledge proofs have already been standardized (respectively in XX and XX), there is no formal, established way to implement sigma protocols and their composition. We stress that this is not straightforward:
%Despite Schnorr proofs have been standardized as zero-knowledge proofs (c.f.\ \href{https://tools.ietf.org/html/rfc8235}{RFC8235}) and signatures (c.f.\ ed25519 in \href{https://tools.ietf.org/html/rfc8032}{RFC8032}),
 for instance, the current standardized ed25519 cannot be immediately adapted to a zero-knowledge proof in a secure way, because of its malleability~\cite[p. 7]{JCEng:BDLSY12g},  and the different behavior on the batched and compressed form\footnote{\url{https://hdevalence.ca/blog/2020-10-04-its-25519am}}.
 Additionally, as academic papers focus on proving the security of sigma protocols in generic prime-order groups, the implementation details are often times overlooked, and, as a result, a lot of insecure implementations have been published in the past. To name a few, well-known pitfalls that have led to insecure implementations:
\begin{itemize}
  \item \textbf{Implementation over non-prime order groups could lead to small subgroup attacks}. The cyclic group where proofs are being implemented must be carefully selected. Implementing sigma protocol over Wirstrass curves is appealing for its performances and no general formulas. However, the presence of a small cofactor could be fatal for security. For instance, in 2017, Monero\footnote{\url{https://www.getmonero.org/2017/05/17/disclosure-of-a-major-bug-in-cryptonote-based-currencies.html}} disclosed a vulnerability in Cryptonote-based e-currencies that would allow double-spending.\href{https://tools.ietf.org/html/rfc2785#ref-LAW}{RFC2785}, that already provides solutions for avoiding small subgroup attacks)
  \item \textbf{Leakage, partial leakage, or reuse of the commitment is fatal.}
  The first message in a sigma protocol, sometimes called nonce or commitment, must be uniformly distributed in order to preserve zero-knowledge. Re-use of the same nonce~\cite{XX}, or partial leakage allows for the complete recovery of the witness to be proven.
  A blatant example of this mistake in the wild was uncovered in 2010, by the hacker group \texttt{fail0verflow}. They showed how SONY was reusing the same nonce for digitally signing Playstation 3 games\footnote{\url{https://media.ccc.de/v/27c3-4087-en-console_hacking_2010}}.  The member could exploit this to calculate the private key, and create valid signatures.

  \item \textbf{A weak impelmentation of Fiat--Shamir heuristic compromises adaptive security.} The second message in a sigma protocol, sometimes called \emph{challenge}, if computed non-interactives must include non only the commitment, but the full statement and the group description. In fact, if the group generator $G$ is not included in the (EC-DSA) signature, then it is possible to prove statement under a different generator $\alpha G$, for any $\alpha \in \ZZ_p$; if $X \in \GG$, part of the statement, is not included in the query to the random oracle, it is possible to compute proofs for $\beta X$, for any $\beta \in \ZZ_p$.
  A mistake of this type was uncovered in 2012, by Bernhard et al.~\cite{AC:BerPerWar12}.
  They demonstrated that Helios, a platform for e-voting, calculated the challenge using the Fiat--Shamir heuristic, without including the generator, allowing for tampering of votes.

\end{itemize}

\section{Notation and Terminology}
%Following [BFS], we consider a deterministic parameter genration algorithm Pgen that returns a group description $\Gamma \defeq (p, G, \GG)$ of a (cyclic) additive, prime order group $\GG = \langle G \rangle$ of order $p$.
For the purpose of this document, the following notation will be used:

\begin{tabular}{r@{\hspace{1em}}p{9cm}}
    $\secpar$ & main security parameter\\
    $\GG = \langle G \rangle$ & cyclic group of prime order $p$ generated by $G$\\
    $\prover,\verifier,\dots$ & potentially randomized algorithms\\
    $x\getsr\someset$ & assignment of a uniformly random element in $\someset$ to $x$\\
    $x\getsr\alg(in)$ & assignment of the ouput of a randomized algorithm $\alg$ on input $in$ to $x$\\
    $\relation$ & binary relation\\
    $\lang(\relation)$ & language induced by a binary relation $\relation$, i.e., $\lang(\relation)=\{y\in\{0,1\}^*:\exists w$ such that $(y,w)\in\relation\}$.\\
    $|y|$ & bitlength of a string\\
    $\NN,\ZZ$ & non-negative natural numbers and integers, respectively
\end{tabular}


Following~\cite{AC:BelFucSca16}, we consider a deterministic parameter generation algorithm $\grgen$ that returns a group description $\Gamma \defeq (p, G, \GG)$ of a (cyclic) additive, prime order group $\GG = \langle G \rangle$ of order $p$. With $\secpar$, we denote the security parameter.



\subsection{Formal Definitions}
  In the following we provide the formal definitions required for the remainder of this document.

\subsubsection{$\Sigma$-Protocols}~\\

In the following we formally describe the class of $\Sigma$-protocols, which cover all protocols considered in the remainder of this document.
 For an in-depth discussion of the underlying theory we refer, e.g., to Cramer~\cite{cramer97}.
\begin{definition}\label{def:sigma}
  Let $\relation$ be a binary relation and let $(y,w)\in\relation$.
  An interactive two-party protocol specified by algorithms $(\prover_1,\prover_2,\verifier)$ is called a \emph{$\Sigma$-protocol} for $\relation$ with challenge set $\CS$, public input $y$, and private input $w$, if and only if it satisfies the following conditions:
  \begin{description}
    \item[\bf 3-move form:]
      The protocol is of the following form (cf. also Protocol~\ref{fig:generic}):
      \begin{itemize}
        \item
          The prover computes $(t,\st) \getsr \prover_1(w,y)$ and sends $t$ to the verifier, while keeping $\st$ secret.
        \item
          The verifier draws $c\getsr\CS$ and returns it to the prover.
        \item
          The prover computes $s\getsr\prover_2(w,y,c,\st)$ and sends $s$ to the verifier.
        \item
          The verifier accepts the protocol run, if and only if $\verifier(y,t,c,s)=\accept$, otherwise it rejects.
      \end{itemize}
      The protocol transcript $(t,c,s)$ is called \emph{accepting} if the verifier accepts the protocol run.
    \item[\bf Completeness:]
      If $(y,w)\in\relation$, then the verifier always outputs $\accept$.
    \item[\bf $t$-special soundness:]
      There exists an efficient algorithm $\extr$ which, given $y$ and $t$ accepting protocol transcripts $(t,c_i,s_i)$ for $i=1,\dots,t$ for public input $y$ with the same first message but pairwise distinct challenges (i.e., $c_i\ne c_j$ for $i\ne j$), returns $w$ such that $(y,w)\in\relation$.
    \item[\bf Special honest-verifier zero-knowledge:]
      There exists an efficient algorithm $\simulator$, which on input $y$ and a challenge $c\in\CS$, outputs transcripts of the form $(t,c,s)$ whose distribution is indinstinguishable from accepting protocol transcripts generated by real protocol runs on public input $y$ and with challenge $c$.
  \end{description}
    \begin{protocol}{Generic flow of a $\Sigma$-protocol\\[-2.25em]}{fig:generic}{t!}
      \begin{tabular}{@{}l@{\hspace{-2em}}c@{\hspace{-2em}}r@{}}
        $\prover[w,y,R]$ & & $\verifier[y,R]$  \\
        \hline  \\
      % -----------------------------P1-------------------------------
        $ (t,\st) \getsr \prover_1(w,y)$\\
        & $\sendr{t}{100}$ \\[2 ex]
      % -----------------------------CHALLENGE-------------------------------
        & & $c \getsr \CS$ \\
        & $\sendl{c}{100}$ & \\[2 ex]
      % -----------------------------P2-------------------------------
        $ s \getsr \prover_2(w,y,c,\st)$\\
        & $\sendr{s}{100}$ \\[2 ex]
      % -----------------------------V-------------------------------
        & & $\accept/\reject \getsr \verifier(y,t,c,s)$ \\
      \end{tabular}
    \end{protocol}
\end{definition}
Note that $\Sigma$-protocols were originally introduced for the case $t=2$ only.
However, we use the above generalized definition as certain practically relevant optimization techniques require $t>2$.

A $\Sigma$-protocol is said to have \emph{unpredictable commitments} if the probability of generating a collision in the first message is negligible, i.e., if there is a negligible function $\negl$ such that for all $(y,w)\in\relation$ it holds that:
$$
  \Pr[t'=t'' : t'\getsr\prover_1(w,y), t''\getsr\prover_1(w,y)] \leq \negl\,.
$$

A $\Sigma$-protocol is furthermore said to have \emph{unique reponses} if it is infeasible to find two distinct valid responses for a given first message and fixed challenge, i.e., there exist no values $y,t,c,s',s''$ with $s'\ne s''$ such that $\verifier(y,t,c,s')=\verifier(y,t,c,s'')=\accept$.

\subsubsection{Proof Systems and Proofs of Knowledge}~\\

The concept of interactive proofs of knowledge was first mentioned by Goldreich et al.~\cite{STOC:GolMicRac85}, and was then refined by Feige et al.~\cite{STOC:FeiFiaSha87}.
The definitions in the following are due to Bellare and Goldreich~\cite{C:BelGol92}.

Intuitively, a proof system is sound, if it is not possible to make the verifier accept for statements for which no valid witness exists, cf. also~\cite[1.6.2]{zkproof-reference}.
\begin{definition}\label{def:soundness}
  Let $\relation$ be a binary relation, $\sigma:\NN\to[0,1]$ and let $\prover$ and $\verifier$ specify a probabilistic interactive protocol, where at least $\verifier$ is polynomial time.
  The protocol is called \emph{sound} with \emph{soundness error} $\sigma$, if for every $y\notin\lang(\relation)$, every $w\in\{0,1\}^*$, and every interactive algorithm $\prover^*$, the probability that $\prover^*$ makes $\verifier$ output $\accept$ on common input $y$ is bounded above by $\sigma(|y|)$.
\end{definition}

Informally, an interactive protocol is a proof of knowledge, if every party that is able to make the verifier accept with sufficiently high probability needs to know a valid witness or would be able to compute such a witness, cf. also~\cite[1.6.3]{zkproof-reference}.
\begin{definition}\label{def:pok}
  Let $\relation$ be a binary relation, $\kappa:\NN\to[0,1]$ and let $\prover$ and $\verifier$ specify a probabilistic interactive protocol, where at least $\verifier$ is polynomial time.
  The protocol is then called a \emph{proof of knowledge} for $\relation$ with \emph{knowledge error} $\kappa$, if the folowing conditions are satisfied:
  \begin{description}
    \item[Completeness:]
      If $(y,w)\in\relation$, then the verifier (on input $y$) always outputs $\accept$ in an interaction with the prover (on input $(y,w)$).
    \item[Knowledge soundness:]
      There exists a probabilistic algorithm $\extr$ (the \emph{knowledge extractor}) and a polynomial $\poly$ such that the following holds:
      For every interactive algorithm $\prover^*$ and every $y\in\lang(\relation)$, let $\varepsilon(y,\prover^*)$ be the probability that $\prover^*$ makes $\verifier$ output $\accept$ on common input $y$.
      If $\varepsilon(y,\prover^*)>\kappa(|y|)$, then $\extr$, having rewindable blackbox access to $\prover^*$, outputs $w'$ satisfying $(y,w')\in\relation$ in an expected number of steps bounded by $\frac{\poly[|y|]}{\varepsilon(y,\prover^*)-\kappa(|y|)}$.
  \end{description}
\end{definition}

\section{Constructions for $\Sigma$-Protocols}
\subsection{Basic $\Sigma$-Protocols in Prime-Order Groups}\label{sec:basicsigma}
A basic $\Sigma$-protocol for the relation
$$\relation=\{((y_1,\dots,y_m),(w_1,\dots,w_n)) : (y_1,\dots,y_m)=\varphi(w_1,\dots,w_n)\}$$
 for a group homomorphism $\varphi:\ZZ_p^n\to\GG^m$ is given by the following algorithms:

\begin{enumerate}
  \item\label{item:basic:p1}
    The prover's first algorithm $\prover_1(w,y)$ consists of the following steps:
    \begin{enumerate}
      \item\label{item:basic:p1:randomness}
        It chooses a random element $r_1,\dots,r_n\getsr\GG$.
      \item
        It then computes $(t_1,\dots,t_m)\gets\varphi(r_1,\dots,r_n)$.
      \item
	The algorithm sets $\st\gets (r_1,\dots,r_n)$ and $t\gets(t_1,\dots,t_m)$.
      \item
        It finally outputs outputs $(t,\st)$.
    \end{enumerate}
  \item\label{item:basic:p2}
    The prover's second algorithm $\prover_2(w,y,c,\st)$ proceeds as follows:
    \begin{enumerate}
      \item
        It checks that $c\in\ZZ_u$ and aborts if this is not the case.
      \item
	It then parses $\st=(r_1,\dots,r_n)$ and $w=(w_1,\dots,w_n)$.
      \item
        It computes its output as $s_i\gets r_i+cw_i$ for $i=1,\dots,n$.
    \end{enumerate}
  \item\label{item:basic:v}
    The verifier's algorithm $\verifier(y,t,c,s)$ proceeds as follow:
    \begin{enumerate}
      \item
        It parses $y=(y_1,\dots,y_m)$, $t=(t_1,\dots,t_m)$, and $s=(s_1,\dots,s_n)$.
      \item\label{item:basic:v:checks}
        It checks that $s_i\in\ZZ_p$ for $i=1,\dots,n$ and $t_j\in\GG$ for $j=1,\dots,m$, and outputs $0$ if this is not the case.
      \item
	It checks whether $(t_1 + cy_1,\dots,t_m + cy_m) = \varphi(s_1,\dots,s_n)$, and outputs $1$ if this is the case; otherwise, $\verifier$ outputs $\reject$.
    \end{enumerate}
  \item\label{item:basic:sim}
    The required simulator $\simulator(y,c)$ for a basic $\Sigma$-protocol works as follows:
    \begin{enumerate}
      \item
        It parses $y=(y_1,\dots,y_m)$.
      \item\label{item:basic:sim:s}
        It chooses $s_1,\dots,s_n\getsr\GG$.
      \item
        It sets $(t_1,\dots,t_m) \gets \varphi(s_1,\dots,s_n) - c(y_1,\dots,y_m)$.
      \item
        Finally, the algorithm ouputs the simulated transcript by setting $(t,c,s)\gets((t_1,\dots,t_m),c,(s_1,\dots,s_n)$.
    \end{enumerate}
\end{enumerate}

\subsubsection{Proving linear relations among witnesses}

  While the above protocol allows one to efficiently prove knowledge of a preimage under a homomorphism, many protocols found in the literature require one to prove relations among witnesses.
  Specifically, they require to prove relations like the following:
\begin{align*}
\relation=\{((y_1,\dots,y_m),(w_1,\dots,w_n)) :& (y_1,\dots,y_m)=\varphi(w_1,\dots,w_n) \\
                                               & A(w_1,\dots,w_n) = (b_1,\dots,b_k)\}\,,
\end{align*}
where the matrix $A\in\ZZ_q^{k\times n}$ and vector $(b_1,\dots,b_k)\in\ZZ_p^k$ specify the system of linear equations.

Proving such a relation can easily by achieved by modifying the above protocol as follows:
\begin{itemize}
  \item
    In step \ref{item:basic:p1:randomness}, the prover draws the randomnesses such that they satisfy the system of equations, i.e., such that $A(r_1,\dots,r_n)=(b_1,\dots,b_k)$.
  \item
    In step \ref{item:basic:v:checks}, the verifier additionally checks that $A(s_1,\dots,s_n)=(c+1)(b_1,\dots,b_k)$ and outputs $\reject$ if this is not the case.
  \item
    In step \ref{item:basic:sim:s}, the simulator draws the responses such at they satisfy the verification equations, i.e., such that $A(s_1,\dots,s_n)=(c+1)(b_1,\dots,b_k)$.
\end{itemize}




\subsubsection{Examples}

\paragraph{Example 1 (DLOG).}
Let $\GG$ be a group over an elliptic curve with prime order $p$.
Proving knowledge of the discrete logarithm $w$ of a point $Y$ in base $G$ means proving knowledge of $w\in\ZZ_p$ such that $Y=wG$.
For a description of this proof goal in residue classes, we also refer to~\cite[1.4.1]{zkproof-reference}.

Using the above notation, we have $\varphi:\ZZ_p\to\GG:x\mapsto xG$.
The protocol flow is then as depicted in Protocol~\ref{fig:dlog}.
    \begin{protocol}{Proving knowledge of a discrete logarithm.\\[-2.25em]}{fig:dlog}{h!}
      \begin{tabular}{@{}l@{\hspace{2em}}c@{\hspace{-3em}}r@{}}
        $\prover[w,Y,G]$ & & $\verifier[Y,G]$  \\
        \hline  \\
      % -----------------------------P1-------------------------------
        $ r\getsr\ZZ_p$ & &\\
        $ T \gets rG$ & & \\
        & $\sendr{T}{100}$ \\[2 ex]
      % -----------------------------CHALLENGE-------------------------------
        & & $c \getsr \ZZ_p$ \\
        & $\sendl{c}{100}$ & \\[2 ex]
      % -----------------------------P2-------------------------------
        $ s \gets r + cw$\\
        & $\sendr{s}{100}$ \\[2 ex]
      % -----------------------------V-------------------------------
        & & Return $1$ if and only if \\
        & & $T + cY = sG$ \\
      \end{tabular}
    \end{protocol}

For a given challenge $c\in\ZZ_p$, the simulator chooses $s\getsr\ZZ_p$, and sets $T\gets sG-cY$.
It then outputs the simulated transcript $(T,c,s)$.


\paragraph{Example 2 (DLEQ).}
Let $\GG$ be a group over an elliptic curve with prime order $p$.
Proving equality of the known discrete logarithm $w$ of $Y_1$ in base $G$ and $Y_2$ in base $H$ means proving knowledge of $(w_1,w_2)\in\ZZ_p$ such that $Y_1=w_1G$ and $Y_2=w_2H$, and $w_1=w_2$.

Using the above notation, we have $\varphi:\ZZ_p\to\GG\times\GG:(x_1,x_2)\mapsto (x_1G,x_2H)$.
The linear system of equations $A(w_1,w_2)=b$ is given by $w_1-w_2=0$.
The protocol flow is then as depicted in Protocol~\ref{fig:dleq}.
    \begin{protocol}{Proving knowledge of equality of two discrete logarithms.\\[-2.25em]}{fig:dleq}{h!}
      \begin{tabular}{@{}l@{\hspace{-4em}}c@{\hspace{-3em}}r@{}}
        $\prover[(w_1,w_2),(Y_1,Y_2),(G,H)]$ & & $\verifier[(Y_1,Y_2),(G,H)]$  \\
        \hline  \\
      % -----------------------------P1-------------------------------
        $ r_1,r_2\getsr\ZZ_p$ such that $r_1-r_2=0$ & &\\
        $ T_1 \gets r_1G$ & & \\
        $ T_2 \gets r_2H$ & & \\
        & $\sendr{T_1,T_2}{100}$ \\[2 ex]
      % -----------------------------CHALLENGE-------------------------------
        & & $c \getsr \ZZ_p$ \\
        & $\sendl{c}{100}$ & \\[2 ex]
      % -----------------------------P2-------------------------------
        $ s_1 \gets r_1 + cw_1$\\
        $ s_2 \gets r_2 + cw_2$\\
        & $\sendr{s_1,s_2}{100}$ \\[2 ex]
      % -----------------------------V-------------------------------
        & & Return $1$ if and only if \\
        & & $T_1 + cY_1 = s_1G$ \\
        & & $T_2 + cY_2 = s_2H$ \\
        & & and $s_1-s_2=0$.
      \end{tabular}
    \end{protocol}

For a given challenge $c\in\ZZ_p$, the simulator chooses $s_1,s_2\getsr\ZZ_p$ such that $s_1-s_2=0$, and sets $T_1\gets s_1G-cY_1$ and $T_2\gets s_2H - cY_2$.
It then outputs the simulated transcript $((T_1,T_2),c,(s_1,s_2))$.

\paragraph{Example 3 (DLEQ; alternative).}
The same proofgoal as in the previous example can also be achieved by considering a slightly different homomorphism which directly encodes the linear relation, that is $\varphi:\ZZ_p\to\GG\times\GG:x\mapsto (xG,xH)$.
The protocol flow is then as depicted in Protocol~\ref{fig:dleq_2}.
    \begin{protocol}{Proving knowledge of equality of two discrete logarithms (alternative).\\[-2.25em]}{fig:dleq_2}{h!}
      \begin{tabular}{@{}l@{\hspace{2em}}c@{\hspace{-3em}}r@{}}
        $\prover[w,(Y_1,Y_2),(G,H)]$ & & $\verifier[(Y_1,Y_2),(G,H)]$  \\
        \hline  \\
      % -----------------------------P1-------------------------------
        $ r\getsr\ZZ_p$ & &\\
        $ T_1 \gets rG$ & & \\
        $ T_2 \gets rH$ & & \\
        & $\sendr{T_1,T_2}{100}$ \\[2 ex]
      % -----------------------------CHALLENGE-------------------------------
        & & $c \getsr \ZZ_p$ \\
        & $\sendl{c}{100}$ & \\[2 ex]
      % -----------------------------P2-------------------------------
        $ s \gets r + cw$\\
        & $\sendr{s}{100}$ \\[2 ex]
      % -----------------------------V-------------------------------
        & & Return $1$ if and only if \\
        & & $T_1 + cY_1 = sG$ \\
        & & and $T_2 + cY_2 = sH$. \\
      \end{tabular}
    \end{protocol}

For a given challenge $c\in\ZZ_p$, the simulator chooses $s\getsr\ZZ_p$, and sets $T_1\gets sG-cY_1$ and $T_2\gets sH - cY_2$.
It then outputs the simulated transcript $((T_1,T_2),c,s)$.

\paragraph{Example 4 (REP).}
Let $\GG$ be a group over an elliptic curve with prime order $p$.
Proving knowledge of a valid opening of a Pedersen commitment means proving knowledge of $w_1,w_2\in\ZZ_p$ such that $Y=w_1G + w_2H$.

Using the above notation, we have $\varphi:\ZZ_p\times\ZZ_p\to\GG:(x_1,x_2)\mapsto x_1G + x_2H$.
The protocol flow is then as depicted in Protocol~\ref{fig:rep}.
    \begin{protocol}{Proving knowledge of representation.\\[-2.25em]}{fig:rep}{h!}
      \begin{tabular}{@{}l@{\hspace{2em}}c@{\hspace{-3em}}r@{}}
        $\prover[(w_1,w_2),Y,(G,H)]$ & & $\verifier[Y,(G,H)]$  \\
        \hline  \\
      % -----------------------------P1-------------------------------
        $ r_1\getsr\ZZ_p$ & &\\
        $ r_2\getsr\ZZ_p$ & &\\
        $ T \gets r_1G + r_2H$ & & \\
        & $\sendr{T}{100}$ \\[2 ex]
      % -----------------------------CHALLENGE-------------------------------
        & & $c \getsr \ZZ_p$ \\
        & $\sendl{c}{100}$ & \\[2 ex]
      % -----------------------------P2-------------------------------
        $ s_1 \gets r_1 + cw_1$\\
        $ s_2 \gets r_2 + cw_2$\\
        & $\sendr{s_1,s_2}{100}$ \\[2 ex]
      % -----------------------------V-------------------------------
        & & Return $1$ if and only if \\
        & & $T + cY = s_1G + s_2H$ \\
      \end{tabular}
    \end{protocol}

For a given challenge $c\in\ZZ_p$, the simulator chooses $s_1,s_2\getsr\ZZ_p$, and sets $T\gets s_1G + s_2H -cY$.
It then outputs the simulated transcript $(T,c,(s_1,s_2))$.

\paragraph{Example 5 (DH).}
Let $\GG$ be a group over an elliptic curve with prime order $p$.
Proving knowledge of the exponents of a valid Diffie-Hellman triple means proving knowledge of $w_1,w_2\in\ZZ_p$ such that $Y_1=w_1G$, $Y_2=w_2G$, and $Y_3=w_1 w_2 G$.
Yet, the mapping $\ZZ_p^2\to\GG^3:(x_1,x_2)\mapsto (x_1G,x_2G,x_1x_2G)$ is not a homomorphism, and consequently the basic protocol presented before cannot be deployed directly.
However, the required multiplicative relation can be proven by observing that the proof goal is equivalent to $Y_1=w_1G$, $Y_2=w_2G$, and $Y_3=w_2Y_1$, leading the homomorphism $\varphi:\ZZ_p^2\to\GG^3:(x_1,x_2)\mapsto(x_1G,x_2G,x_2Y_1)$.

The protocol flow is then as depicted in Protocol~\ref{fig:dh}.
    \begin{protocol}{Proving knowledge of representation.\\[-2.25em]}{fig:dh}{h!}
      \begin{tabular}{@{}l@{\hspace{-3em}}c@{\hspace{-2em}}r@{}}
        $\prover[(w_1,w_2),(Y_1,Y_2,Y_3),G)]$ & & $\verifier[(Y_1,Y_2,Y_3),G]$  \\
        \hline  \\
      % -----------------------------P1-------------------------------
        $ r_1\getsr\ZZ_p$ & &\\
        $ r_2\getsr\ZZ_p$ & &\\
        $ T_1 \gets r_1G$ & & \\
        $ T_2 \gets r_2G$ & & \\
        $ T_3 \gets r_2Y_1$ & & \\
        & $\sendr{T_1,T_2,T_3}{100}$ \\[2 ex]
      % -----------------------------CHALLENGE-------------------------------
        & & $c \getsr \ZZ_p$ \\
        & $\sendl{c}{100}$ & \\[2 ex]
      % -----------------------------P2-------------------------------
        $ s_1 \gets r_1 + cw_1$\\
        $ s_2 \gets r_2 + cw_2$\\
        & $\sendr{s_1,s_2}{100}$ \\[2 ex]
      % -----------------------------V-------------------------------
        & & Return $1$ if and only if \\
        & & $T_1 + cY_1 = s_1$ \\
        & & $T_2 + cY_2 = s_2$ \\
        & & and $T_3 + cY_3 = s_2Y_1$ \\
      \end{tabular}
    \end{protocol}

For a given challenge $c\in\ZZ_p$, the simulator chooses $s_1,s_2\getsr\ZZ_p$, and sets $T_1\gets s_1G -cY_1$, $T_2\gets s_2G-cY_2$, and $T_3\gets s_2Y_1-cY_3$.
It then outputs the simulated transcript $((T_1,T_2,T_3),c,(s_1,s_2))$.

As shown in this example, and in contrast to linear relations, multiplicative relations among witnesses typically require a reformulation of the proof goal in order to be compatible with the generic protocol presented above.
We refer, e.g., to Krenn~\cite{krenn12} for generic techniques.

\subsection{Composition of $\Sigma$-Protocols}\label{sec:composition}
  In this section, we recap composition techniques of $\Sigma$-protocols.
  Specifically, we define mechanisms for proving knowledge of multiple independent witnesses (``AND-composition''), and for proving knowledge for one out of a set of witnesses (``OR-composition'').
  Without loss of generality, the techniques presented in the following focus on the composition of two protocols;
  proving knowledge of more than two witnesses, or for out of a larger set of witnesses, can directly be achieved by iteratively deploying the techniques.

  For the rest of this section, we let $(\prover_1^0,\prover_2^0,\verifier^0)$ and $(\prover_1^1,\prover_2^1,\verifier^1)$ be the specifications of two $\Sigma$-protocols for two relations $\relation^0$ and $\relation^1$, and let $\simulator^0$ and $\simulator^1$ be their simulators.

  Furthermore, we assume that the challenge sets are given by $\ZZ_{u^0}$ and $\ZZ_{u^1}$, respectively.
  We let $u=\min(u^0,u^1)$.


\subsubsection{AND Composition.}
  In the following we explain how to construct a $\Sigma$-protocol proving knowledge of multiple independent witnesses.
  That is, the algorithms specified below constitute a $\Sigma$-protocol for the following relation:
$$
  \relation^\land = \left\{ ((y^0,y^1),(w^0,w^1) : (y^0,w^0)\in \relation^0 ~\land~ (y^1,w^1)\in\relation^1\right\}\,.
$$


\begin{enumerate}
  \item
    The prover's first algorithm $\prover_1(w,y)$ consists of the following steps:
    \begin{enumerate}
      \item
        The algorithm parses $w=(w^0,w^1)$ and $y=(y^0,y^1)$.
      \item
        It computes $(t^0,\st^0)\getsr\prover_1^0(y^0,w^0)$ and $(t^1,\st^1)\getsr\prover_1^1(y^1,w^1)$.
      \item
	The algorithm outputs $(t,\st) = ((t^0,t^1),(\st^0,\st^1))$.
    \end{enumerate}
  \item
    The prover's second algorithm $\prover_2(w,y,c,\st)$ proceeds as follows:
    \begin{enumerate}
      \item
        It checks that $c\in\ZZ_u$ and aborts if this is not the case.
      \item
	It parses $\st=(\st^0,\st^1)$
      \item
        It computes $s^0\getsr\prover_2^0(w^0,y^0,c,\st^0)$ and $s^1\getsr\prover_2^1(w^1,y^1,c,\st^1)$.
      \item
        It outputs $s\gets(s^0,s^1)$.
    \end{enumerate}
  \item
    The verifier's algorithm $\verifier(y,t,c,s)$ proceeds as follow:
    \begin{enumerate}
      \item
        It  parses $s=(s^0,s^1)$.
      \item
	The algorithm outputs $\verifier^0(y^0,t^0,c,s^0)\land\verifier^1(y^1,t^1,c,s^1)$.
    \end{enumerate}
  \item
    The required simulator $\simulator(y,c)$ works as follows:
    \begin{enumerate}
      \item
        It parses $y=(y^0,y^1)$.
      \item
        It chooses $c\in\ZZ_u$.
      \item
        It computes $(t^0,c,s^0)\getsr\simulator^0(y^0,c)$ and $(t^1,c,s^1)\getsr\simulator^1(y^1,c)$.
      \item
        Finally, the algorithm then outputs $(t,c,s)=((t^0,t^1),c,(s^0,s^1))$.
    \end{enumerate}
\end{enumerate}


\subsubsection{OR Composition.}
  In the following we explain how to construct a $\Sigma$-protocol proving knowledge of one out of a set of witnesses.
  That is, the algorithms specified below constitute a $\Sigma$-protocol for the following relation:
$$
  \relation^\lor = \left\{ ((y^0,y^1),(w^0,w^1) : (y^0,w^0)\in \relation^0 ~\lor~ (y^1,w^1)\in\relation^1\right\}\,.
$$

  In the following protocol specification, let $j$ be such that $w^j$ is known to the prover, whereas without loss of generality $w^{1-j}$ is assumed to be unknown to the prover.
\begin{enumerate}
  \item
    The prover's first algorithm $\prover_1(w,y)$ consists of the following steps:
    \begin{enumerate}
      \item
        The algorithm parses $w=(w^0,w^1)$ and $y=(y^0,y^1)$, where $w^{1-j}=\bot$.
      \item
        It computes $(t^j,\st^j)\getsr\prover_1^j(y^j,w^j)$.
      \item
        It computes a simulated transcript for the unknown witness by choosing $c^{1-j}\getsr\ZZ_u$ and setting $(t^{1-j},c^{1-j},s^{1-j})\getsr\simulator^{1-j}(y^{1-j},c^{1-j})$.
      \item
	The algorithm outputs $(t,\st) = ((t^0,t^1),(\st^j,c^{1-j},s^{1-j}))$.
    \end{enumerate}
  \item
    The prover's second algorithm $\prover_2(w,y,c,\st)$ proceeds as follows:
    \begin{enumerate}
      \item
        It checks that $c\in\ZZ_u$ and aborts if this is not the case.
      \item
	It parses $\st=(\st^j,c^{1-j},s^{1-j})$.
      \item
        It computes $c^j\gets c-c^{1-j}\mod m$, and sets $s^j\getsr\prover_2^j(w^j,y^j,c^j,\st^j)$.
      \item
        It computes $s^0\getsr\prover_2^0(w^0,y^0,c,\st^0)$ and $s^1\getsr\prover_2^1(w^1,y^1,c,\st^1)$.
      \item
        It outputs $s\gets(s^0,s^1,c^0)$.
    \end{enumerate}
  \item
    The verifier's algorithm $\verifier(y,t,c,s)$ proceeds as follow:
    \begin{enumerate}
      \item
        It  parses $s=(s^0,s^1,c^0)$.
      \item
        It sets $c^1\gets c-c^0\mod m$.
      \item
	The algorithm outputs $\verifier^0(y^0,t^0,c^0,s^0)\land\verifier^1(y^1,t^1,c^1,s^1)$.
    \end{enumerate}
  \item
    The required simulator $\simulator(y,c)$ works as follows:
    \begin{enumerate}
      \item
        It parses $y=(y^0,y^1)$.
      \item
        It chooses a random $c^0$ in $\ZZ_u$ and computes $c^1\gets c-c^0 \mod m$.
      \item
        It then computes $(t^0,c^0,s^0)\getsr\simulator^0(y^0,c^0)$ and $(t^1,c^1,s^1)\getsr\simulator^1(y^1,c^1)$.
      \item
        Finally, the algorithm then outputs $(t,c,s)=((t^0,t^1),c,(s^0,s^1,c^0))$.
    \end{enumerate}
\end{enumerate}


\subsection{Achieving Non-Interactivity -- The Fiat-Shamir Transform}
All protocols describes so far require three messages being exchanged between the prover and the verifier.
However, communication rounds are often considered expensive from an efficiency point of view, and for many applications interactivy is not desirable.

The following two variants tp achieve non-interactivity require identical computations on the prover's side.
The first variant has a larger proof size, as the full first message of the prover is part of the proof, whereas the second variant only has a hash value instead.
However, the first version allows for \stnote{why again do we want to have the long form?}

The constructions are based on the seminal work of Fiat and Shamir~\cite{C:FiaSha86} and subsequent work, e.g., by Bernhard et al.~\cite{AC:BerPerWar12}.
The underlying idea of this so-called Fiat-Shamir transform is to simulate the verifier's random challenge by means of a random oracle, depending on the first message computed by the prover.
More precisely, the challenge is computed as $c\gets\hash(t,y,\ctx)$, where $t$ is the prover's first message from the $\Sigma$-protocol, and $y$ is the public image for which knowledge of a witness is proven.
The context string $\ctx$ contains application-specific information, such as:
\begin{itemize}
  \item
    a full description of the algebraic setting and proof goal (e.g., group descriptions, generators, etc.) to achieve non-malleability;
  \item
    local context information (e.g., session identifiers of higher level protocols) to avoid replay-attacks, or shared randomness or a timestamp to guarantee freshness of the proof.
\end{itemize}

\subsubsection{Long form}


\begin{enumerate}
  \item
    The prover's algorithm $\prover'(y,w,\ctx)$ works as follows:
    \begin{enumerate}
      \item
        The algorithm first computes $(t,\st)\getsr\prover_1(y,w)$.
      \item
        It computes the challenge by setting $c\gets\hash(t,y,\ctx)$.
      \item
        The algorithm defines $s\getsr\prover_2(y,w,c,\st)$.
      \item
        The algorithm outputs $(t,s)$.
    \end{enumerate}
  \item
    The verifier's algorithm $\verifier'(y,t,s,\ctx)$ works as follows:
    \begin{enumerate}
      \item
        It recomputes the challenge as $c\gets\hash(t,y,\ctx)$.
      \item
        It outputs whatever $\verifier(y,t,c,s)$ outputs.
    \end{enumerate}
\end{enumerate}

\subsubsection{Short form}
\begin{enumerate}
  \item
    The prover's algorithm $\prover''(y,w,\ctx)$ works as follows:
    \begin{enumerate}
      \item
        The algorithm first computes $(t,\st)\getsr\prover_1(y,w)$.
      \item
        It computes the challenge by setting $c\gets\hash(t,y,\ctx)$.
      \item
        The algorithm defines $s\getsr\prover_2(y,w,c,\st)$.
      \item
        The algorithm outputs $(c,s)$.
    \end{enumerate}
  \item
    The verifier's algorithm $\verifier''(y,c,s,\ctx)$ works as follows:
    \begin{enumerate}
      \item\label{item:fslong:v:recomputet}
        The algorithm recomputes the first message $t$ by running the appropriate steps of the simulator $\simulator$.
      \item
        The algorithm checks that $c=\hash(t,y,\ctx)$ and rejects if this is not the case.
    \end{enumerate}
\end{enumerate}

More specifically, when calling $\simulator$ in Step~\ref{item:fslong:v:recomputet}, the verifier does not draw the reponse $s$ at random, as is the case in all the protocols introduced before;
instead it uses the value specified in the proof.

\section{Security Considerations}
\minote{mention simulation extractability? XXX}
\minote{mention post-quantum security guarantees?}

In the following, we give a concise overview of the most important security guarantees provided by the constructions presented above.

Firstly, all the protocols presented in the previous section are $\Sigma$-protocols according to Definition~\ref{def:sigma}.
\begin{theorem}
  If $\GG$ is a cyclic group of prime order $p$, $\varphi:\ZZ_p^n\to\GG^m$ is a group homomorphism, and $u\leq p$, then the basic protocol in Section~\ref{sec:basicsigma} is a $\Sigma$-protocol with simulator $\simulator$.

  If $\varphi$ is non-trivial, the protocol has unpredictable commitments.
  Finally, if $\varphi(x')\ne\varphi(x'')$ for all $x'\ne x''$, the protocol has unique responses.
\end{theorem}

\begin{theorem}
  If, for $i=0,1$, $(\prover_1^i,\prover_2^i,\verifier^i)$ is a $\Sigma$-protocol for relation $\relation^i$ with challenge set $\ZZ_{u^i}$ and simulator $\simulator^i$, then the protocols specified in Section~\ref{sec:composition} are $\Sigma$-protocol with challenge set $\ZZ_u$ and simulator $\simulator$ for relation $\relation^\land$ and $\relation^\lor$, respectively.

  It at least one of the composed protocols has unpredictable commitments, then the composed protocol has unpredictable commitments.
%  If all composed protocols have unique responses, then so has the composed protocol.
\end{theorem}


In a nutshell, $\Sigma$-protocols are honest-verifier zero-knowledge proofs of knowledge, whose security guarantees are maintained under the Fiat-Shamir transform at least against classical attackers.









For full details and proofs, we refer to the original literature.
\begin{itemize}
  \item
    If $\prover_1,\prover_2,\verifier$ specify a $\Sigma$-protocol for $\relation$ according to Definition~\ref{def:sigma}, then they also specify an honest-verifier zero-knowledge proof of knowledge with soundness error $\sigma=1/m$ and knowledge error $\kappa=1/m$ for $\relation$.

For formal proofs we refer, e.g., to Damg\r{a}rd~\cite{damgard04}.
  \item
    If a $\Sigma$-protocol is complete and has unpredictable commitments, then the Fiat-Shamir transform is {\bf complete}.

    For a proof we refer to Unruh~\cite{AC:Unruh17}, which also gives a surprising counter-example in case that commitments are not unpredictable.
  \item
    If a $\Sigma$-protocol is honest-verifier zero-knowledge and has unpredictable commitments, then the Fiat-Shamir transform yields a {\bf zero-knowledge} protocol in the random oracle model.

    For a proof we refer to Faust et al.~\cite{INDOCRYPT:FKMV12} and Unruh~\cite{AC:Unruh17}.
  \item
    If a $\Sigma$-protocol has a negligible soundness error, then the Fiat-Shamir transform is {\bf sound} according to Definition~\ref{def:soundness}.
    This security guarantee also holds against quantum attackers.

    For a proof we refer to Pointcheval and Stern~\cite{JC:PoiSte00} and Unruh~\cite{AC:Unruh17}.
  \item
    If a $\Sigma$-protocol has a negligible soundness error, then the Fiat-Shamir transform is {\bf weakly simulation-sound} according to Definition~\ref{def:xxx}.
    If the protocol additionally has unique responses, then the Fiat-Shamir transform is {\bf strongly simulation-sound} according to Definition~\ref{def:xxx}.
    These security guarantees also holds against quantum attackers.

    For a proof we refer to Unruh~\cite{AC:Unruh17} and Faust et al.~\cite{INDOCRYPT:FKMV12}.
  \item
    If a $\Sigma$-protocol has a negligible soundness error, then the Fiat-Shamir transform is {\bf simulation-sound extractable} according to Definition~\ref{def:xxx}.

    For a proof we refer to Bernhard et al.~\cite{AC:BerPerWar12}.
\end{itemize}

\section{Implementation}
Implementations available + selecting the ellptic curves + performance table + proof size.

\subsection{Pre-existing implementations}

supported statement + latest version with date + benchmarks + creation date + citation.
\paragraph{SCAPI.}
\paragraph{zkp.}
\paragraph{zksk.}
\paragraph{Emmy}
\paragraph{Helios}
\subsection{A candidate API design}

\paragraph{Key Generation.}

\paragraph{Nonce Generation.} The first step is the generation of a uniformly-distributed random element over $\ZZ_p$. As mentioned in \Cref{sec:motivation}, the nonce must be distributed uniformly at random, and even a small bias in the distribution could completely compromise zero-knowledge.
The construction of cryptographically secure source of randomness is a difficult problem, that is particularly challenging on embedded devices such as smart-cards or embedded systems.
For this reason, we propose the generation of a synthetic nonce obtained from hashing the statement and the secret together, as already suggested in previous standards \href{https://tools.ietf.org/html/rfc6979}{RFC6979}.

\minote{XXX. what do we do for ring signatures that repeat the same statement twice? suggest to add a counter or a new random entry in the message?}

\stnote{Maybe we should explicitly say here that this way to generate the nonce is only possible for non-interactive protocols, and that in the interactive case this would allow the verifier to use different challenges for the same first message, allowing the verifier to extract}

\paragraph{Computing the challenge.} just 128 bits expect post-quantum security + use blake2 or sha256 truncated or sha512 read as an element of $\ZZ_p$


\section{To be discussed}
If the proposal is accepted, we would discuss the following points in the working group.
\minote{designated verifier proofs: either or proof I'm you or the staement is true + commitment discrete log}
\minote{delayed input}
\minote{batching proofs}
\section{Open Problems}
\minote{check multisignatures and add the new problem of computing proofs with devices that share different keys}
\minote{R1CS compatibility: a unified way of specifying proof goals. Cam-Sta notation ? Compilers}

\bibliographystyle{alpha}
\bibliography{cryptobib/abbrev3,cryptobib/crypto,additional}
%
\end{document}
