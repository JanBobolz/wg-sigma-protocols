% This is samplepaper.tex, a sample chapter demonstrating the
% LLNCS macro package for Springer Computer Science proceedings;
% Version 2.20 of 2017/10/04
%
\documentclass[runningheads]{llncs}
%
\usepackage{amsmath}
\usepackage{amssymb}
%\usepackage{amsthm}
\usepackage{mathtools}
%
\usepackage{todonotes}
%
\usepackage{hyperref}
\usepackage{cleveref}
%
\usepackage{graphicx}
% Used for displaying a sample figure. If possible, figure files should
% be included in EPS format.
%
% If you use the hyperref package, please uncomment the following line
% to display URLs in blue roman font according to Springer's eBook style:
\renewcommand\UrlFont{\color{blue}\rmfamily}

\newcommand{\HH}{\mathcal{H}}

\newcommand{\getsr}{\gets_{\$}}

\newcommand{\minote}[1]{\todo[color=green!30, inline]{\textbf{Michele's note:} #1}}

\begin{document}

\title{Proposal: $\Sigma$-protocols%
\thanks{This work has partially been funded by the European Union's Horizon 2020 framework programme under grant agreement no. 830929 (CyberSec4Europe).}}
%
\titlerunning{Proposal: $\Sigma$-protocols}

\author{Stephan Krenn\orcidID{0000-0003-2835-9093} \and
        Michele Orr\`u}

\authorrunning{S. Krenn and M. Orr\`u}

\institute{AIT Austrian Institute of Technology, Vienna, Austria \and
           University of California, Berkeley, United States}
%
\maketitle              % typeset the header of the contribution
%
\begin{abstract}
The abstract should briefly summarize the contents of the paper in
150--250 words.

\keywords{First keyword  \and Second keyword \and Another keyword.}
\end{abstract}

\section{Introduction}
\subsection{Related Work}

\section{Background and Motivation}

\section{Notation and Terminology}

\section{Constructions for $\Sigma$-Protocols}
\subsection{Basic $\Sigma$-Protocols in Prime-Order Groups}



\subsection{Composition of $\Sigma$-Protocols}
\subsubsection{AND Composition}

\subsubsection{OR Composition}

\subsection{Achieving Non-Interactivity - The Fiat-Shamir Transform}


\section{Security Considerations}

\section{Implementation}

\bibliographystyle{splncs04}
\bibliography{cryptobib/abbrev3,cryptobib/crypto}
%
\end{document}
